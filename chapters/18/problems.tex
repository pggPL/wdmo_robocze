\begin{problem}{1}
	W pewnej grupie $2n$ osób, gdzie $n$ jest liczbą całkowitą nie mniejszą od $2$, każda osoba zna co najmniej $n$ innych osób. Wykazać, że możemy usadzić pewne cztery osoby przy okrągłym stole, aby każda z nich znała swoich sąsiadów.
\end{problem}

\begin{problem}{2}
	Dana jest pewna dodatnia liczba całkowita $k$ oraz pewna liczba chłopców i dziewczyn. Każda z osób zna dokładnie $k$ osób płci przeciwnej. Wykazać, że można zaaranżować pary między chłopcami i dziewczynami, tak, aby każda osoba była w dokładnie jednej parze.
\end{problem}

\begin{problem}{3}
	Dany jest graf, w którym każdy wierzchołek ma stopień co najwyżej $100$. Zbiór krawędzi nazwiemy \textit{idealnym}, jeśli żadne dwie krawędzie należące do niego nie mają wspólnego końca oraz nie da się dołożyć żadnej krawędzi, aby ta własność była zachowana. W każdym ruchu usunięto dowolny  idealny zbiór krawędzi, które przed wykonaniem tego ruchu należały do rozpatrywanego grafu. Wykazać, że po wykonaniu dowolnych $199$ ruchów, graf nie będzie zawierał żadnej krawędzi.
\end{problem}

\begin{problem}{4}
	Dany jest graf planarny, w którym $e$ to liczba krawędzi, a $v \geqslant 3$ to liczba wierzchołków. Wykazać, że zachodzi nierówność
	\[
		e \leqslant 3v - 6.
	\]
\end{problem}

\begin{problem}{5}
	Na pewnym przyjęciu okazało się, że każdy zna co najmniej $k$ innych gości, gdzie $k \geqslant 2$ jest pewną liczbą naturalną. Wykazać, że istnieje taka liczba naturalna $n\geqslant k + 1$, że można usadzić pewnych $n$ uczestników przyjęcia przy okrągłym stole tak, aby każdy znał obu swoich sąsiadów.
\end{problem}

\begin{problem}{6}
	Krawędzi grafu pełnego o $n \geqslant 3$ wierzchołkach zostały pokolorowane w taki sposób, że każdy kolor został użyty do pokolorowania co najwyżej $n - 2$ krawędzi. Wykazać, że istnieje podgraf tego grafu, będący trójkątem, o krawędziach parami różnych kolorów.
\end{problem}
