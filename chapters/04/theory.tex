% Rozdział 4 – liczby pierwsze i reszty z dzielenia



\theory{Liczby pierwsze i reszty z dzielenia}

\noindent
Ten rozdział będzie nieco bardziej teoretyczny niż poprzednie. Zadania także będą trudniejsze -- zachęcamy do skorzystania ze wskazówek i głębokiego przestudiowania rozwiązań. Chcemy wyrobić u czytelnika intuicję dotyczącą działań na resztach z dzielenia przez pewną liczbę pierwszą. Od czytelniczki/czytelnika wymaga się, aby znał własności kongruencji -- opisano je chociażby w  \href{https://omj.edu.pl/uploads/attachments/kwadrat-07-kolor.pdf}{Gazetce OMJ ,,Kwadrat'' nr 7}. 

\vspace{5px}

\noindent
We wszystkich poniższych zadaniach przez $a$, $b$ będziemy oznaczać liczby całkowite, zaś przez $p$ dowolną liczbę pierwszą. Przez $x \big| y$ będziemy oznaczać fakt, że liczba $x$ jest dzielnikiem liczby $y$.

\vspace{5px}
\heading{Twierdzenie 1}

\noindent
Jeśli liczba $ab$ jest podzielna przez $p$, to wówczas co najmniej jedna z liczb $a$, $b$ jest podzielna przez $p$.

\vspace{15px}

\noindent
Zauważmy, że założenie o pierwszości liczby $p$ jest konieczne. Chociażby liczba $4 \cdot 9 = 36$ dzieli się przez $6$, ale żadna z liczb $4$, $9$ nie jest podzielna przez $6$.

\vspace{5px}
\noindent
Zachęcamy do samodzielnej próby wykazania poniższych lematów. Poniżej, czcionką odwróconą, zapisano wskazówki.

\vspace{5px}
\heading{Lemat 1}

\noindent
Udowodnić, że jeśli $x^2\equiv 1 \pmod{p}$ dla pewnej liczby pierwszej $p$, to \[   
    x\equiv 1 \pmod{p} \quad \text{lub} \quad x\equiv -1 \pmod{p}.
\]

\rotatehint{Zapisz założenia i tezę zadania bez użycia modulo.}

\heading{Dowód}

\noindent
Zauważamy, że zapis $x^2\equiv 1 \pmod{p}$ jest równoważny zapisowi 
\[
    p \mid x^2-1=(x-1)(x+1).
\]
 Skoro $p$ jest liczbą pierwszą, to na mocy Twierdzenia 1 zachodzi $p\mid x-1$ lub $p\mid x+1$, a to jest równoważne temu, co było do wykazania.
 
\qed

\vspace{10px}

\heading{Lemat 2}

\noindent
Liczba $a$ nie jest podzielna przez $p$. Udowodnić, że istnieje taka dodatnia liczba całkowita~$k$, że 
\[
   a^k\equiv 1 \pmod{p}.
\]

\rotatehint{Udowodnij, że istnieją takie $r$ i $s$, że $a^r\equiv a^s \pmod{p}$.}

\heading{Dowód}

\noindent
Rozpatrzmy ciąg $(1,\; a^1,\; a^2,\; a^3,\; ...)$ . Zauważamy, że ma on nieskończenie wiele elementów, a reszt z dzielenia przez $p$ jest skończenie wiele. Z Zasady Szufladkowej Dirichleta mamy więc, że istnieją takie liczby $r$ oraz $s$ -- załóżmy, że $r\geqslant s $ -- że 
\[
    a^r\equiv a^s \pmod{p}.
\]
Jest to równoważne temu, że 
\[  
    p\mid a^s(a^{r-s}-1).
\]
Skoro $a$ nie jest podzielna przez $p$, to $p\mid a^{r-s}-1$, skąd zachodzi $a^{r-s}\equiv 1 \pmod{p}$.

\qed


\vspace{10 px}
\heading{Odwrotności modulo p}

\noindent
Z Lematu 2 można wywnioskować, że dla każdej liczby $a$, która nie jest podzielna przez $p$ istnieje pewna liczba $b \in \{1, 2, ..., p-1\}$, że
\[
    ab \equiv 1 \pmod{p}.
\]
Wystarczy wziąć $b = a^{k - 1} \text{ mod } p$. 

\vspace{10px}
\noindent
Wykażemy teraz, że w zbiorze $\{1, 2, ..., p-1\}$ jest dokładnie jedna taka liczba $b$. Załóżmy, że dla pewnych $b, c \in \{1, 2, ..., p-1\}$
\[
    ab \equiv  ac \equiv 1 \pmod{p}.
\]
Wówczas
\[
    p \big| ab - ac = a(b - c) \implies p \big|b - c,
\]
gdyż liczba $a$ nie jest podzielna przez $p$. Skoro $b, c \in \{1, 2, ..., p-1\}$, to 
\[
    - p < b - c < p.
\]
Skoro $p \big| b - c$, to $b - c = 0$, czyli $b = c$.

\vspace{10px}

\noindent
Przyjmiemy, że liczba $b$ jest \textit{odwrotnością} liczby $a$ modulo $p$. Zapiszemy $b = a^{-1} \pmod{p}$.

\vspace{10px}

\heading{Lemat 3}

\noindent
Dla dowolnej liczby $a$, która nie jest podzielna przez $p$ ciąg
\[
    (a \text{ (mod } p),\;\; 2a \text{ (mod } p),\;\;  3a \text{ (mod } p), ...,\;\;  (p - 1) \cdot a \text{ (mod } p))
\]
jest permutacją ciągu
\[
 (1,\; 2,\; 3,\; ...,\; p - 1).
\]

\rotatehint{Wykaż, że $ai \not\equiv aj \pmod{p}$, jeśli $i \not\equiv j \pmod{p}$.}

\heading{Dowód}

\noindent
Pokażmy, że jeśli $i \not\equiv j \pmod{p}$, to $ai \not\equiv aj \pmod{p}$. 
Załóżmy, że 
\[
    ai \equiv aj \pmod{p}
\] 
dla pewnych $i,\; j$. Skoro $p$ nie dzieli $a$, to istnieje odwrotność $a$ modulo $p$. Mnożąc obie strony przez $a^{-1}$ -- lub równoważnie dzieląc przez $a$ otrzymujemy
\[
    i \equiv j \pmod{p},
\]
co dowodzi postulowanej implikacji.

\vspace{10px}
\noindent
Rozpatrzmy liczby $a, \; 2a,\; (p - 1)a$. Oczywiście żadna z nich nie jest podzielna przez~$p$. Z tego, że tych liczb jest $p - 1$, niezerowych reszt z dzielenia przez $p$ również jest $p - 1$, oraz te liczby dają parami różne reszty niezerowe z dzielenia przez $p$, wynika teza.

\qed


\newpage

\heading{Małe twierdzenie Fermata}

\noindent
Dana jest liczba $a$, która nie jest podzielna przez $p$. Wykazać, że
\[
    a^{p - 1} \equiv 1 \pmod{p}.
\]

\heading{Dowód}

\noindent
Korzystając z poprzedniego lematu mamy, że ciąg
\[
    (a \text{ (mod } p),\;\; 2a \text{ (mod } p),\;\;  3a \text{ (mod } p), ...,\;\;  (p - 1) \cdot a \text{ (mod } p))
\]
jest permutacją ciągu
\[
 (1,\; 2,\; 3,\; ...,\; p - 1).
\]
Skoro te ciągi zawierają te same elementy modulo $p$, tylko, że w innej kolejności, to iloczyny tych elementów będą dawały taką samą resztę z dzielenie przez $p$. Więc
\[
    1 \cdot 2 \cdot 3 \cdot ... \cdot (p - 1) = a \cdot 2a \cdot 3a \cdot ... \cdot (p - 1)a \pmod{p},
\]
\[
    (p - 1)! \equiv a^{p - 1}(p - 1)! \pmod{p}.
\]
Zauważmy, że $(p - 1)! \not\equiv 0 \pmod{p}$. Mnożać przystawanie stronami przez odwrotność liczby~$(p - 1)!$ otrzymujemy tezę.

\qed

\vspace{10px}
