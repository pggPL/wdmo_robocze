\hints{Szachownice, klocki i kolorowanie}

\begin{hints_list}
	\item Zauważ, że jeśli pewien klocek $3\times3$ leży w wierszach $4$, $5$ i $6$, to w każdym z tych wierszy musi istnieć jeszcze jedno pole zajmowane przez klocek $3 \times 3$.
	\item Wykaż, że punkty $X$ i $X'$ są tego samego koloru. Przeprowadź podobne rozumowanie dla tego samego $X$, ale innego $X'$.
	\item *
	\item Na początku ten obwód wynosi co najwyżej $4 \cdot 9$, a docelowo ma wynosić $40$.
	\item Rozpatrz punkty takie jak na rysunku, przy czym załóż, że $X_1$ i $X_2$ są tego samego koloru.
	\begin{center}
	\begin{tikzpicture}[]
		\tkzDefPoint[label = above:$Y_1$](0,0){Y_1}
		\tkzDefPoint[label = above:$X_1$](2,0){X_1}
		\tkzDefPoint[label = above:$M$](3,0){M}
		\tkzDefPoint[label = above:$X_2$](4,0){X_2}
		\tkzDefPoint[label = above:$Y_2$](6,0){Y_2}

		\tkzDrawPoints(X_1, Y_1, X_2, Y_2, M)
		\tkzDrawSegments(Y_1,Y_2)
	\end{tikzpicture}
\end{center}
	\item Zauważ, że prostokąty o czterech czarnych polach narożnych zawierają o jedno pole czarne więcej. Co jeśli prostokąt ma czterech białe pola narożne? Albo dwa białe i dwa czarne pola narożne?
\end{hints_list}