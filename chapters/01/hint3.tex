\newpage
\hints{Indukcja matematyczna}


\begin{hints_list}
	\item Skorzystaj z faktu, że suma kątów w trójkącie wynosi $180\degree$ oraz z założenia indukcyjnego.

	\item *

	\item *

	\item Zauważ, że jeśli z usuniętego punktu wychodzą np. tylko czerwone odcinki, to pomiędzy dowolnymi dwoma punktami da się przejść odcinkami czerwonymi.

	\item Wykaż tezę indukcyjnie za pomocą założenia i udowodnionej równości. Zauważ, że 
	\begin{align*}
		a_0a_1a_2...a_na_{n+1}\left(\frac{1}{a_0} + \frac{1}{a_1} + \frac{1}{a_2} + ... + \frac{1}{a_n} + \frac{1}{a_{n + 1}}\right) = \\ =   a_0a_1a_2...a_na_{n+1}\left(\frac{1}{a_0} + \frac{1}{a_1} + \frac{1}{a_2} + ... + \frac{1}{a_n}\right) + a_0a_1a_2...a_n.
	\end{align*}

	\item Podziel plansze na 1 L-klocek i cztery części, które można pokryć na mocy założenia indukcyjnego.

	\item Zauważ, że suma liczb
	\[
		{x_{2k+3}-x_{2k+1}, \; x_{2k+3}-x_{2k},\; \cdots, \; x_{2k+3}-x_{k+2}} \quad \text{i} \quad {x_{2k+2}-x_{k+1},\; \cdots, \; x_{2k+2}-x_1},
	\]
	jest równa sumie liczb
\begin{gather*}
	x_{2k+3} - x_{2k+2} \quad \text{i} \quad x_{2k+2} - x_{2k+1},\; x_{2k+2} - x_{2k},\; \cdots,\; x_{2k+3} - x_{k+2}  \\ \text{oraz} \quad {x_{2k+3} - x_{k+1},\; \cdots, \; x_{2k+3}-x_1}.
\end{gather*}
\end{hints_list}