% Rozdział 19 - nierówności z pomysłem

\theory{Nierówności z pomysłem}

\noindent
Wiele nierówności da się rozwiązać za pomocą sztampowych technik -- omówiono niektóre z nich w poprzednich rozdziałach. Teraz zajmiemy się nierównościami, do których rozwiązania potrzebna jest nie tyle wiedza, co wpadniecie na ciekawy pomysł. 

\vspace{10px}


\heading{Długości boków w trójkącie}

\noindent
W zadaniach z nierównościami czasami pojawia się założenie, że liczby $a$, $b$ i $c$ są bokami trójkąta. Innymi słowy
\[
	a + b \geqslant c, \quad b + c \geqslant a, \quad c + a \geqslant b.
\]
Jest to bardzo niewygodne założenie, niemniej jednak istnieje pewien trik, który pozwala na znaczne jego uproszczenie. Rozpatrzmy liczby
\[
	x = \frac{a + b - c}{2}, \quad y = \frac{a + c - b}{2}, \quad z = \frac{b + c - a}{2}.
\]
Za mocy założeń wiemy, że są one dodatnie. Mamy też, że
\[
	a = x + y, \quad b = x + z, \quad c = y + z.
\]
Można także wykazać, że jeśli istnieją liczby dodatnie $x$, $y$, $z$, dla których zachodza powyższe równości, to $a$, $b$ i $c$ są długościami boków trójkąta. Istotnie
\begin{align*}
	a + b = (x + y) + (x + z) &\geqslant (y + z) = c, \\
	b + c = (x + z) + (y + z) &\geqslant (x + y) = a, \\
	c + a = (y + z) + (x + y) &\geqslant (x + z) = b.
\end{align*}
Można także wykazać, że odcinki o długości $x$, $y$, $z$ to są długości odcinków stycznych do okręgu wpisanego w trójkącie o bokach długości~$a$, $b$ i~$c$.

\begin{center}
	\begin{tikzpicture}
		\tkzDefPoint(0,0){A};
		\tkzDefPoint(4,0){B};
		\tkzDefPoint(3,1.5){C};

		\tkzDefCircle[in](A,B,C) \tkzGetPoint{I}\tkzGetLength{rABpt}
		\tkzDrawCircle[in](A,B,C)
		\tkzDefPointBy[projection=onto B--C](I)\tkzGetPoint{D}
		\tkzDefPointBy[projection=onto C--A](I)\tkzGetPoint{E}
		\tkzDefPointBy[projection=onto A--B](I)\tkzGetPoint{F}

		\tkzLabelSegment[above left](A,E){$x$}
		\tkzLabelSegment[below](A,F){$x$}
		\tkzLabelSegment[above right](B,D){$y$}
		\tkzLabelSegment[below](B,F){$y$}
		\tkzLabelSegment[above right](C,D){$z$}
		\tkzLabelSegment[above left](C,E){$z$}


		\tkzDrawPoints(A,B,C,D,E,F);
		\tkzDrawSegments(A,B B,C C,A);
	\end{tikzpicture}
\end{center}

\heading{Przykład 1}

\noindent
Dany jest trójkąt o bokach długości $a$, $b$ i $c$. Wykazać, że zachodzi nierówność
\[
	\frac{1}{-a + b + c} + \frac{1}{a - b + c} + \frac{1}{a + b - c} \geqslant \frac{1}{a} + \frac{1}{b} + \frac{1}{c}.
\]

\heading{Rozwiązanie}

\noindent
Wykazaliśmy, że warunek zadania jest równoważny temu, że istnieją takie dodatnie liczby $x$, $y$, $z$, dla których
\[
	a = x + y, \quad b = x + z, \quad c = y + x.
\]
Przepisując postulowaną nierówność za pomocą powyższych podstawień otrzymujemy
\begin{align*}
	\frac{1}{-(x + y) + (x + z) + (y + z)} &= \frac{1}{2z}, \\
	\frac{1}{(x + y) - (x + z) + (y + z)} &= \frac{1}{2y}, \\
	\frac{1}{(x + y) + (x + z) - (y + z)} &= \frac{1}{2x}. \\
\end{align*}
Trzeba więc wykazać, że
\begin{align*}
	\frac{1}{2z} + \frac{1}{2y} + \frac{1}{2x} \geqslant \frac{1}{x + y} + \frac{1}{x + z} + \frac{1}{y + z} \\
	\left(\frac{1}{x} + \frac{1}{y}\right) + \left(\frac{1}{x} + \frac{1}{z}\right) + \left(\frac{1}{y} + \frac{1}{z}\right) \geqslant \frac{4}{x + y} + \frac{4}{x + z} + \frac{4}{y + z}.
\end{align*}
Zauważmy, że nierówność
\[
	\frac{1}{x} + \frac{1}{y} \geqslant \frac{4}{x + y}
\]
wynika wprost z nierówności między średnią arytmetyczną i harmoniczną. Dodając analogicznie nierówności stronami otrzymujemy tezę.

\qed

\heading{Przykład 2}

\noindent
Liczby $a$, $b$, $c$ i $d$ spełniają równość $abcd = 1$. Wykazać, że
\[
	\frac{1}{(1 + a)^2} + \frac{1}{(1 + b)^2} + \frac{1}{(1 + c)^2} + \frac{1}{(1 + d)^2} \geqslant 1.
\]

\heading{Rozwiązanie}

\noindent
Pomysłem, który należy rozważyć przy tego typu zadaniach są szacowania pomocnicze. Wykażemy, że
\[
	\frac{1}{(1 + a)^2} + \frac{1}{(1 + b)^2} \geqslant \frac{1}{1 + ab},
\]
dla dowolnych liczb rzeczywistych $a$, $b$. Mamy
\begin{align*}
	\frac{1}{(1 + a)^2} + \frac{1}{(1 + b)^2} &\geqslant \frac{1}{1 + ab}, \\
	(ab + 1)\left( (1 + a)^2 + (1 + b)^2\right) &\geqslant (a + 1)^2(b + 1)^2, 
\end{align*}
co po wymnożeniu i redukcji wyrazów podobnych skróci się do postaci
\begin{align*}
	1 + a^3b + ab^3 &\geqslant 2ab + a^2b^2 \\
	ab(a - b)^2 + (ab - 1)^2 &\geqslant 0.
\end{align*}

\noindent
Korzystając z wykazanego szacowania mamy
\[
	\frac{1}{(1 + a)^2} + \frac{1}{(1 + b)^2} + \frac{1}{(1 + c)^2} + \frac{1}{(1 + d)^2} \geqslant \frac{1}{1 + ab} + \frac{1}{1 + cd} = \frac1{1 + ab} + \frac{ab}{ab + abcd} = 1.
\]
\qed

% może jakaś nierówność z podstawieniem