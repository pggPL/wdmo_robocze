\newpage
\solutions{Ciągi}

\begin{problem}{1}
	Dana jest liczba pierwsza $p>3$. Wykazać, że istnieje taki ciąg arytmetyczny dodatnich liczb całkowitych $\{x_i\}$, że $x_1x_2\cdot ... \cdot x_p$ jest kwadratem liczby całkowitej.
\end{problem}

\noindent 
Rozpatrzmy ciąg $x_i=ip!$.
Wówczas
\[
	\prod_{i=1}^p x_i=\prod_{i=1}^p ip!=(p!)^{p}\cdot p! = (p!)^{p}\cdot p!=(p!)^{p+1}.
\]
co oczywiście jest kwadratem liczby całkowitej.
\\


\begin{problem}{2}
	Dany jest ciąg $a_1, a_2, ..., a_{2n + 1}$ liczb rzeczywistych, że dla dowolnych liczb całkowitych ${2n + 1 \geqslant i, j \geqslant 1}$ zachodzi równość
	\[
		a_i + a_j \geqslant |i - j|.
	\]
	Wyznaczyć najmniejszą możliwą sumę elementów tego ciągu.
\end{problem}

\answer{Najmniejsza możliwa suma elementów tego ciągu wynosi $n(n + 1)$}

\noindent
Na mocy założenia zadania zachodzą równości
\begin{align*}
	a_1 + a_{2n + 1} \geqslant 2n, \\
	a_2 + a_{2n} \geqslant 2n - 2, \\
	..., \\
	a_n + a_{n + 2} \geqslant 2. \\
\end{align*}
Wiemy również, że $a_{n + 1} + a_{n + 1} \geqslant 0$, czyli liczba $a_{n + 1}$ jest nieujemna. Dodając powyższe nierówności stronami otrzymujemy
\[
	a_1 + a_2 + ... + a_{2n + 1} \geqslant 2 + 4 + ... + (2n - 2) + 2n = 2(1 + 2 + ... + n) = n(n + 1).
\]
Wykażemy, że ciąg dany wzorem
\[
	a_k = |n + 1 - k|
\]
spełnia warunki zadania. Istotnie, na mocy nierówności trójkąta
\[
	a_i + a_j = |n + 1 - i| + |n + 1 - j| \geqslant |(n + 1 - j) - (n + 1 - i)| = |i - j|.
\]
Zauważmy, że
\[
	a_1 + a_2 + ... + a_{2n + 1} = n + (n - 1) + ... + 1 + 0 + 1 + ... + (n - 1) + n = n(n + 1),
\]
co dowodzi tego, że postulowane minimum jest osiągalne.


\begin{problem}{3}
	Ciąg ${u_n}$ dany jest wzorem $u_0=1, u_{n+1}=\frac{u_n}{u_n+3}$. Wykaż, że $ u_1+u_2+...+u_{2021}<1$
\end{problem}

\noindent
Przekształcając równanie rekurencyjnie równoważnie otrzymujemy
\begin{align*}
	u_{n + 1} &= \frac{u_n}{u_n + 3}, \\
	\frac{1}{u_{n + 1}} &= \frac{u_n + 3}{u_n} = 1 + \frac{3}{u_n}.
\end{align*}
Podstawmy $a_n = \frac{1}{u_n}$. Wówczas
\[
	a_0 = 1, \; a_{n + 1} = 3a_n + 1. 
\]
Sprawdzenie wartości $a_n$ dla małych $n$ pozwala postawić hipotezę, że $a_n = \frac{3^{n + 1} - 1}{2}$. Łatwo sprawdzić, że dla $a_0$ postulowana równość zachodzi oraz z tożsamości
\[
	\frac{3^{n + 2} - 1}{2} = 3 \cdot \frac{3^{n + 1} - 1}{2} + 1
\]
wynika, że jeśli postulowana równość zachodzi dla $a_n$, to zachodzi również dla $a_{n + 1}$. Z zasady indukcji matematycznej możemy więc wywnioskować, że
\begin{align*}
	a_{n} &= \frac{3^{n + 1} - 1}{2}, \\
	u_n &= \frac{1}{a_n} = \frac{2}{3^{n + 1} - 1}.
\end{align*}
Mamy więc
\[
	u_1 + u_2 + ... + u_{2021} < \sum^{\infty}_{i = 0} \frac{2}{3^{i + 1} - 1} < \sum^{\infty}_{i = 0} \frac{2}{3^{i + 1}} = \frac{2}{3} \cdot \sum^{\infty}_{i = 0} \frac{1}{3^{i}} = \frac{2}{3} \cdot \frac{3}{2} = 1.
\]

\begin{problem}{4}
	Znajdź wszystkie liczby rzeczywiste $x$, że ciąg dany wzorem
	\[
		a_0 = x, \quad a_n = 2^n - 3a_n,
	\]
	dla dowolnej liczby naturalnej n spełnia zależność
	\[
		a_{n} > a_{n - 1}.
	\]
\end{problem}

\noindent
Zauważmy, że
\begin{align*}
	a_{n + 1} &= 2^n - 3a_n = 2^n - 3(2^{n - 1} - 3a_{n - 1}) = 2^n - 3 \cdot 2^{n - 1} + 9a_{n - 1} =  \\
	&= 2^n - 3 \cdot 2^{n - 1} + 9(2^{n - 2} - 3a_{n - 2}) = 2^n - 3 \cdot 2^{n - 1} + 9 \cdot 2^{n - 2} - 27a_{n - 2} = \\
	&= \sum^{n}_{i = 0} \left(2^{n - i} \cdot (-3)^i\right) + (-3)^{n + 1}a_0 = 2^n \sum^{n}_{i = 0} \left(-\frac{3}{2}\right)^i + (-3)^{n + 1}a_0 = \\
	&= 2^n \cdot \frac{1 - \left(-\frac{3}{2}\right)^{n + 1}}{1 + \frac{3}{2}} + (-3)^{n + 1}a_0 = \frac{2^{n + 1}}{5} +\left(a_0 - \frac{1}{5}\right)(-3)^{n + 1}.
\end{align*}
Jeśli $a_0 \neq \frac{1}{5}$, to liczba $\left(a_0 - \frac{1}{5}\right)(-3)^{n + 1}$ będzie przyjmowała dowolnie duże i dowolnie małe wartości. Ich wartość bezwzględna będzie odpowiednio dużych $n$ rzędy większa niż $\frac{2^{n + 1}}{5}$. Toteż $a_n$ również będzie przyjmowała dowolnie małe wartości, toteż ciąg $a_n$ nie może być rosnący. Jeśli zaś $a_0 = \frac{1}{5}$, to z powyższej tożsamości wynika, że ta liczba spełnia warunki zadania.

\vspace{5px}

\begin{problem}{5}
	Dany jest ciąg
	\[
		a_0 = 1, \quad a_{n + 1} = a_n + \frac{1}{a_n}.
	\]
	Rozstrzygnąć, czy $a_{5000} > 100$.
\end{problem}

\answer{Zachodzi nierówność $a_{5000} > 100$.}

\noindent
Przekształćmy równanie rekurencyjne ciągu
\begin{align*}
	a_{n + 1} &= a_n + \frac{1}{a_n}, \\
	a_{n + 1}^2 &= \left(a_n + \frac{1}{a_n}\right)^2 = a_n^2 + 2 a_n \cdot \frac{1}{a_n} + \frac{1}{a_n^2}  > a_n^2 + 2.
\end{align*}
Z powyższej nierówności wynika, że
\[
	a_{5000}^2 > a_{4999}^2 + 2 > a_{4998}^2 + 4 > ... > a_0^2 + 10000 > 10000, 
\]
z~czego wynika, że $a_{5000}^2 > 100$.

\vspace{5px}

\begin{problem}{6}
	Wyznaczyć wszystkie dodatnie liczby rzeczywiste $\alpha$, dla których istnieje ciąg dodatnich liczb rzeczywistych $x_1$, $x_2$, $x_3$, $\dots$ o tej własności, że dla wszystkich $n\geqslant 1$:
	\[
		x_{n+2} = \sqrt{\alpha x_{n+1} - x_n}.
	\]
\end{problem}

\answer{Szukany ciąg istnieje wtedy i tylko wtedy, gdy $\alpha > 1$.}

\noindent
Jeśli $\alpha > 1$, to łatwo sprawdzić, że ciąg dany wzorem $x_n  = \alpha - 1$ spełnia warunki zadania. Załóżmy nie wprost, że dla pewnego $\alpha \leqslant 1$ istnieje ciąg spełniający warunki zadania. Wówczas, skoro liczba pod pierwiastkiem musi być dodatnia, mamy
\begin{align*}
 	0 &\leqslant \alpha x_{n+1} - x_n \leqslant x_{n + 1} - x_n, \\
 	x_n &\leqslant x_{n + 1},
\end{align*}
z czego wynika, że ciąg ten jest rosnący. Wykażemy teraz, że od pewnego momentu ciąg ten przyjmuje wartości mniejsze niż $1$. Istotnie
\[
	x_{n + 2}^2 = \alpha x_{n + 1} - x_n < \alpha x_{n + 1} \leqslant x_{n + 1} \leqslant x_{n + 2},
\]
z czego wynika postulowana własność.
Zauważmy, że
\begin{align*}
	x_1 < x_{n + 2} = \sqrt{\alpha x_{n+1} - x_n} \leqslant \sqrt{x_{n+1} - x_n}, \\
	x_1^2 < x_{n+1} - x_n.
\end{align*}
Z powyższej nierówności, jak i z wykazanej wyżej własności, wynika, że
\[
	1 > x_{n + 1} > x_n + x_{1}^2 > x_{n - 1} + 2x_{1}^2 > ... > nx_{1}^2,
\]
co nie może być prawdą dla wszystkich liczb naturalnych $n$.

\vspace{5px}

\begin{problem}{7}
	Dany jest ciąg
	\[
		a_0 = 6, \quad a_n = a_{n - 1} + NWD(n, a_{n - 1}).
	\]
	Wykazać, że $NWD(n, a_{n - 1})$ jest liczbą pierwszą lub jest równe 1.
\end{problem}


\noindent
Kluczowe w rozwiązaniu zadania jest następujące umocnienie tezy.

\vspace{5px}

\noindent
Jeśli dla pewnej liczby całkowitej~$n$ zachodzi nierówność $\mathrm{NWD}(n, a_{n - 1}) > 1$, to $a_n = 3n$ oraz $\mathrm{NWD}(a_{n − 1}, n)$ jest liczbą pierwszą. 

\vspace{10px}
\noindent
Wyjściowo ciąg przyjmuje wartości
\[
	a_1 = 7, \; a_2 = 8, \; a_3 = 9, \; a_4 = 10, \; a_5 = 15.
\] 
Możemy więc zauważyć, że dla $n = 5$ mamy 
\[
	\mathrm{NWD}(a_{n − 1}, n) = 5 \;\; \text{oraz} \;\; a_5 = 3 \cdot 5.
\]
 Wykażemy tezę indukcyjnie -- zakładamy, że zachodzi ona dla wszystkich liczb mniejszych od $n$. Jeśli $NWD(n, a_n - 1) = 1$, to nic nie trzeba wykazywać. Załóżmy więc, że $NWD(n, a_n - 1) > 1$.  Weźmy największą liczbę $t$, mniejszą niż $n$, że $\mathrm{NWD}(t, a_t − 1) > 1$. Na mocy założenia indukcyjnego mamy $a_t = 3t.$
 Wówczas 
 \[
 	\mathrm{NWD}(i, a_i − 1) = 1 \text{ dla wszystkich liczb } i \in \{t + 1, t + 2, t + 3, ..., n - 1\},
 \] 
czyli $a_{i} = a_{i - 1} + 1$. Stąd
\[
	a_n − 1 = a_t + (n − t) − 1 = 3t + (n - t) - 1 =  2t + n − 1.
\]
Zatem 
\[
	\mathrm{NWD}(a_n − 1,\; n) = \mathrm{NWD}(2t + n - 1,\; n) = \mathrm{NWD}(2t - 1,\; n) = \mathrm{NWD}(2t - 1,\; 2n - (2t - 1)).
\] 
Liczba $n$ jest najmniejszą liczbą większą od $t$, dla której wartość powyższego wyrażenia jest większa od $1$. Nietrudno zauważyć, że pierwszą liczbą większą od $t$, dla której jest to prawda, jest takie $n$, że liczba $2n - (2t - 1)$ jest najmniejszym nieparzystym dzielnikiem pierwszym liczby $2t - 1$. Wówczas oczywiście $2n - (2t - 1)$ jest liczbą pierwszą oraz
\begin{align*}
	\mathrm{NWD}(n, a_n − 1) &= 2n - (2t - 1), \\
	a_n &= a_{n - 1} + \mathrm{NWD}(n, a_n − 1) = (2t + n − 1 ) + 2n - (2t - 1) = 3n,
\end{align*}
co kończy dowód indukcyjny.

