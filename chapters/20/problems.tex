\begin{problem}{1}
	Wykazać, że dla dowolnych liczb całkowitych $a$, $b$, równanie
	\[
		(x^2 - y^2 - a)(x^2 - y^2 - b)(x^2 - y^2 - ab) = 0
	\]
	ma przynajmniej jedno rozwiązanie w liczbach całkowitych $x$, $y$.
\end{problem}

\begin{problem}{2}
	Dana jest liczba całkowita $n > 1$, dla której liczba $2^{2^n - 1} - 1$ jest liczbą pierwszą. Wykazać, że $n$ również jest liczbą pierwszą.
\end{problem}

\begin{problem}{3}
	Dane są dodatnie liczby całkowite $a$, $b$ i $c$, takie że
	\[
		b \mid a^3, \quad c \mid b^3, \quad a \mid c^3.
	\]
	Wykazać, że
	\[
		abc \mid (a + b + c)^{13}.
	\]
\end{problem}

\begin{problem}{4}
	Niech $x$, $y$, $z$, $t$ będą dodatnimi liczbami całkowitymi, takimi, że zachodzi równość
	\[
		x^2 + y^2 + z^2 + t^2 = 2000!.
	\]
	Wykazać, że każda z liczb $x$, $y$, $z$, $t$ jest większa niż $10^{200}$.
\end{problem}

\begin{problem}{5}
	Niech $n$ będzie liczbą całkowitą, a $p$ liczbą pierwszą. Wiadomo, że liczba $np + 1$ jest kwadratem liczby całkowitej. Wykazać, że $n + 1$ można zapisać za pomocą sumy $p$ kwadratów liczb całkowitych.
\end{problem}


\begin{problem}{6}
	(a) Wykazać, że dla dowolnej liczby całkowitej $m$ istnieje taka liczba całkowita $n\geqslant m$, że
\[
	\left \lfloor \frac{n}{1} \right \rfloor \cdot \left \lfloor \frac{n}{2} \right \rfloor \cdots \left \lfloor \frac{n}{m} \right \rfloor = \binom{n}{m} \\\\\\\\\\\\\\\        (*)
\]
(b) Niech $p(m)$ będzie najmniejszą liczbą $n \geqslant m$, że zachodzi równanie  $(*)$. Udowodnić, że $p(2018) = p(2019).$
\end{problem}







