\hints{Podzielności}

\begin{hints_list}
	\item Zauważ, że $(k + 1)^2 > n \geqslant k^2$.
	\item Skorzystaj z nierówności między średnią arytmetyczną i geometryczną dla dwóch liczb.
	\item Wykaż, że istnieje nieskończenie wiele trójek liczb $(k, k + 1, k + 2)$, że każda liczba z tej trójki jest dzielnikiem liczby $4n$.
	\item Wykaż, że $b - 1 \mid a + 1$. Połącz dwie otrzymane podzielności.
	\item Załóżmy, że $a > b$. Jeśli $a$ i $b$ dają tę samą resztę z dzielenia przez $p$, to $a \geqslant b + p$.
	\item Zauważ, że $a$ i $b^k$ mają inne $v_p$. Nasuwa to skojarzenie z Lematem 4.
\end{hints_list}