\hints{Gry}

\begin{hints_list}
	\item Zawsze da się uzyskać $1$. Wykazać, że da się, być może w kilku ruchach, zmniejszyć zapisaną na tablicy liczbę.
	\item Analizuj pola tabeli $m \times n$ jako stany wygrywające i przegrywające.
	\item Wykaż, że liczby parzyste są wygrywające.
	\item Podziel planszę na pewne figury. Rozważ strategię, która każde drugiemu graczowi stawiać swój znak w tej samej figurze, w której postawił go gracz pierwszy.
	\item Niech dla pewnej liczby $n$ w pierwszych $4n$ ruchach nauczyciel narysuje kwadrat o boku $n$.
	\item Przyjmijmy, że na każdym kamieniu są miejsca na zapisanie dwóch liczb. Jeśli pewna dziewczyna $a$ przekaże dziewczynie $b$ kamień po raz pierwszy, to o ile na kamieniu nie jest nic zapisane, to zapisuje się tam numery $a$ i $b$.  Jeśli pewna dziewczyna musi przekazać kamień swojej sąsiadce, to jeśli dysponuje kamieniem z zapisanymi numerami swoimi i sąsiadki, to przekaże jej właśnie go. Zobacz, co z tego wynika.
\end{hints_list}